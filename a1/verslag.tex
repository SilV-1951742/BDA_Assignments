% Created 2021-10-10 Sun 18:03
% Intended LaTeX compiler: pdflatex
\documentclass[11pt]{article}
\usepackage[utf8]{inputenc}
\usepackage[T1]{fontenc}
\usepackage{graphicx}
\usepackage{grffile}
\usepackage{longtable}
\usepackage{wrapfig}
\usepackage{rotating}
\usepackage[normalem]{ulem}
\usepackage{amsmath}
\usepackage{textcomp}
\usepackage{amssymb}
\usepackage{capt-of}
\usepackage{hyperref}
\usepackage[T1]{fontenc}
\usepackage[sfdefault]{biolinum}
\usepackage[activate={true,nocompatibility},final,tracking=true,kerning=true,spacing=true,factor=1100,stretch=10,shrink=10]{microtype}
\author{Sil Vaes en Maarten Evenepoel}
\date{\today}
\title{Verslag BDA opdracht 1}
\hypersetup{
 pdfauthor={Sil Vaes en Maarten Evenepoel},
 pdftitle={Verslag BDA opdracht 1},
 pdfkeywords={},
 pdfsubject={},
 pdfcreator={Emacs 29.0.50 (Org mode 9.4.6)}, 
 pdflang={English}}
\begin{document}

\maketitle
\setlength\parindent{0pt}

In deze opdracht was de bedoeling dat we een maximal frequent itemset algoritme ging implementeren. Hier hebben we het a-priori algoritme met hashing, dus PCY.\\


Het inlezen van de dataset in het XML-formaat wordt gedaan met behulp van regexes, dus de XML library in de standard library van Python is niet gebruikt. We hebben bewust deze keuze gemaakt omdat dit ons eenvoudiger leek, want we moesten enkel de entries tussen bepaalde tags matchen en dan enkel the authors. Dus we zijn tot de conclusie gekomen dat een hele XML parser wat overkill was.\\


De dataset wordt ook niet in een keer ingelezen, maar in chunks van \emph{x} megabytes, waar \emph{x} een commandline argument is.
\end{document}