% Created 2021-10-10 Sun 18:03
% Intended LaTeX compiler: pdflatex
\documentclass[11pt]{article}
\usepackage[utf8]{inputenc}
\usepackage[T1]{fontenc}
\usepackage{graphicx}
\usepackage{grffile}
\usepackage{longtable}
\usepackage{wrapfig}
\usepackage{rotating}
\usepackage[normalem]{ulem}
\usepackage{amsmath}
\usepackage{textcomp}
\usepackage{amssymb}
\usepackage{capt-of}
\usepackage{hyperref}
\usepackage[T1]{fontenc}
\usepackage[sfdefault]{biolinum}
\usepackage[activate={true,nocompatibility},final,tracking=true,kerning=true,spacing=true,factor=1100,stretch=10,shrink=10]{microtype}
\author{Sil Vaes en Maarten Evenepoel}
\date{\today}
\title{Verslag BDA opdracht 2}
\hypersetup{
 pdfauthor={Sil Vaes en Maarten Evenepoel},
 pdftitle={Verslag BDA opdracht 2},
 pdfkeywords={},
 pdfsubject={},
 pdfcreator={Emacs 29.0.50 (Org mode 9.4.6)}, 
 pdflang={English}}
\begin{document}

\maketitle
\setlength\parindent{0pt}

Voor dit project was het de bedoeling om de topic drift binnen een gegeven wetenschappelijk onderzoeksveld doorheen de tijd te bestuderen. Wij hebben ervoor gekozen om voor deze opdracht te topic drift binnen het onderzoeksveld rond Data Mining te bestuderen. 

\section{Aanpak}

Om dit probleem op te lossen hebben we gebruik gemaakt van het k-means clustering algoritme. Het idee is om artikels uit de brondata samen te clusteren wanneer deze op elkaar gelijken op basis van de titels van de artikels. Een cluster stelt in dit geval een verzameling artikels voor die op elkaar gelijken, en dus wellicht over eenzelfde of gelijkaardig topic gaan. Als we dit dan doen door artikels in periodes van 15 jaar apart te beschouwen en te clusteren, krijgen we een overzicht van de verschillende topics die voorkomen in elke periode van 15 jaar. Met deze informatie kunnen we vervolgens dan ook de topic drift observeren van topics binnen data mining doorheen de tijd. \\

Vooraleer we effectief data kunnen clusteren is er wat pre-processing nodig bij het inladen van de data. We gaan immers niet werken met de gehele dblp.xml dataset, maar enkel met de Data Mining gerelateerde artikels. De eerste stap die we toepassen is dan ook het extracten van enkel de data mining gerelateerde artikels uit de brondata. Dit doen we door de artikels te filteren op hun key, en te controleren of de key een van de volgende values bevat, zoals gegeven in de opgave: KDD, PKDD, ICDM, SDM. \\

Vervolgens voeren we een feature extraction uit op de titels van de overblijvende documenten met behulp van een \hyperlink{https://scikit-learn.org/stable/modules/generated/sklearn.feature_extraction.text.TfidfVectorizer.html}{TFIDF vectorizer}. De vectorizer die we gebruiken returned voor elk artikel een matrix aan features behorende tot het betreffende artikel. Dit wil zeggen dat elk artikel dus wordt voorgesteld door een matrix aan numerieke vectors of die het belang van verschillende gemeten features in de titels voorstelt. Het k-means algoritme dat we hierop volgend gebruiken kan deze features gebruiken om de onderlinge afstand tussen artikels tijdens het clusteren te helpen bepalen. \\

Wanneer we gebruik maken van k-means moeten we op voorhand twee parameters beslissen, namelijk welke afstandsdefinitie we gebruiken, en in hoeveel clusters we de data willen laten opdelen door het algoritme. In onze implementatie maken we gebruiken van de standaard afstandsdefinitie die ingebakken zit in de kmeans module van sklearn, dus de euclidische afstand tussen de feature matrices die de verschillende artikel titels voorstellen. Daarnaast moeten we ook beslissen in hoeveel clusters we de data willen opdelen. Hiervoor bestaan verschillende methoden.
Wij hebben ervoor gekozen om dit manueel met het ``Elbow Criterion'' te doen. Soms waren deze grafieken redelijk onduidelijk, zeker bij het interval van 1960 tot 1975. Dit komt waarschijnlijk door het kleine aantal artikels in het interval. Ook hebben we geprobeerd het DBSCAN algoritme te gebruiken. Om resultaten te verkrijgen en om te guessen hoeveel cluster centers we zouden hebben. Na het algoritme wat te gebruiken en te spelen met de parameters kregen we hier geen goede resultaten uit. Uit de visualitie na reductie met PCA en t-SNE kregen we geen goede clusters en de top artikels om de topics te bepalen is ook moeilijk met DBSCAN omdat er niet echt een notie is van afstand. Dus hebben we besloten deze straat te verlaten en gewoon verder te gaan met het ``Elbow Criterion''.

% Wij hebben ervoor gekozen het DBSCAN algoritme te gebruiken om het juiste aantal clusters voor elke decade die we gaan clusteren te bepalen. 
% TODO: DBSCAN procedure hier uitleggen

Tot slot kunnen we overgaan tot het effectief clusteren van data. Zoals eerder gezegd beschouwen we de inputdata in periodes van 15 jaar. Ook laten we de verschillende periodes die we beschouwen overlappen met een overlap van 5 jaar. Concreet gaan ge we dus data clusteren over volgende periodes: 1960-1975, 1970-1975, ..., 2010-2025. Per tijdesperiode berekenen we dan zoals eerder gezegd de nodige aantal clusters met het DBSCAN algoritme en voeren we kmeans op de data in de betreffende tijdsperiode uit. 

\section{Resultaten en Output}

We hebben ons programma uitgevoerd op de gehele DBLP dataset. Verder is het programma uitgevoerd op een machine met een Intel Core i5-4690K@4.3GHz in combinatie met 16GB DD3 Dual Channel memory. Het gebruikte besturingssysteem was Ubuntu 21.10. Overigens zagen we dat het piek geheugenverbruik stijgt tot slechts 1GB. Dit komt omdat we de DBLP dataset verwerken in chunks van 1GB, en de selectie artikels die over Data Mining gaan kleiner is dan 1GB. De gemeten runtime van het programma bedraagt 2 min en 12 sec. De output die gegeven wordt is telkens het aantal clusters per tijdsperiode, en enkele representatieve artikels binnen elke cluster binnen de bijhorende tijdsperiodes. De bekomen output is de volgende:

\begin{verbatim}
Clusters in range of year 1960 - 1975
95 titles in this range
Top terms per cluster 1960 - 1975:
Cluster 0: "Program Portability, Data Manipulation Languages, and Data Description Languages." - "Data Description for Data Independence." - "Another Look at Data-Bases." - "A Taxonomy of Data Definition Languages." - "Virtual Information in Data-Base Systems." - 
Cluster 1: "SIGFIDET Bibliography." - "SIGFIDET Bibliography: Second Spasm." - "Bibliography: Data Definition and Description." - 
Cluster 2: "Report on Standards." - "Reports on Standards." - "A Report on the CODASYL Meeting." - "External Liaison Activities Report: The GUIDE Information Management Group." - "Report on SIGFIDET Meeting, SJCC, 1972." - 
Cluster 3: "Correction." - "Correction and Announcements." - 
Cluster 4: "FDT Forum." - "FDT Forum." - 
Cluster 5: "Understanding Relations (Installment #4)." - "Understanding Relations (Third Installment)." - "Understanding Relations (Installment #5)." - "Understanding Relations." - "Understanding Relations." - 
Cluster 6: "Availability Notices." - "Availability Notices." - "Availability Notices." - 
Cluster 7: "Call for Papers: 1972 ACM SIGFIDET Workshop on Data Description, Access and Control." - "1971 ACM-SIGFIDET Workshop on Data Description, Access and Control, Call for Papers" - "Call for Papers: ACM SIGFIDET Workshop on Data Description, Access and Control 1974." - "1971 ACM SIGFIDET Workshop on Data Description, Access and Control - Program." - "Final Program for 1972 ACM-SIGFIDET Workshop on Data Description, Access and Control." - 


Clusters in range of year 1970 - 1985
348 titles in this range
Top terms per cluster 1970 - 1985:
Cluster 0: "SIGMOD Regional Workshops." - "Proposals Sought for SIGMOD Regional Conferences or Workshops." - "Proposals Sought for SIGMOD Regional Conferences or Workshops." - "ACM SIGMOD Annual Business Meeting, 1975." - "ACM SIGMOD Annual Business Meeting, 1974." - 
Cluster 1: "An Informal Approach to Formal Specifications." - "An Informal Approach to Formal Specifications." - "Research Issues in Database Specification." - "Data Flow Structures for System Specification and Implementation." - "Formal Data Base Specification - An Eclectic Perspective." - 
Cluster 2: "The Temporal Query Language TQuel." - "Distributed Query Processing." - "Interfacing a Query Language to a CODASYL DBMS." - "Some Principles of Good Language Design (with especial reference to the design of database languages)." - "Summary-Table-By-Example: A Database Query Language for Manipulating Summary Data." - 
Cluster 3: "Call for Papers: 1972 ACM SIGFIDET Workshop on Data Description, Access and Control." - "Final Program for 1972 ACM-SIGFIDET Workshop on Data Description, Access and Control." - "1971 ACM SIGFIDET Workshop on Data Description, Access and Control - Program." - "1971 ACM-SIGFIDET Workshop on Data Description, Access and Control, Call for Papers" - "Call for Papers: ACM SIGFIDET Workshop on Data Description, Access and Control 1974." - 
Cluster 4: "Understanding Relations (Installment #4)." - "Understanding Relations (Installment #5)." - "Understanding Relations (Installment #6)." - "Understanding Relations (Installment #7)." - "Understanding Relations (Third Installment)." - 
Cluster 5: "On Some Metrics for Databases or What is a Very Large Database?" - "Report on the Workshop on Statistical Database Management." - "How Baroque Should a Statistical Database Management System Be?" - "Metadata Management for Large Statistical Databases." - "The Theory of Relational Databases." - 
Cluster 6: "A Natural Language Interface for Performing Database Updates." - "A Transportable Natural Language Database Update System." - "On Some Arguable Claims in B. Shneiderman's Evaluation of Natural Language Interaction with Database Systems." - "Why Sort-Merge Gives the Best Implementation of the Natural Join." - "The SIGFIDET Annual Business Meeting, August 28, 1973." - 
Cluster 7: "Comments on "Note on the Expected Size of a Join"." - "Note on the Expected Size of a Join." - "Notes on the State of Data Base Audit." - "Editor's Note." - 
Cluster 8: "Research Directions in Data Base Management Systems." - "Data Base Technical Conferences." - "Administering a Distributed Data Base Management System." - "The Plethora of Data Base Conferences." - "Common Interface between Programming Languages abd Data Base Management Systems / Structured Programming and Integrated Data/Base Systems Design." - 
Cluster 9: "A Role for Data Analysis in Practical Requirements Definition." - "Program Analysis for Conversion from a Navigation to a Specification Database Interface." - "Representing Roles in Universal Scheme Interfaces." - "The GENISYS Data Definition Facilities." - "The Devolution of Functional Analysis." - 
Cluster 10: "Notes from the Vice Chairperson." - "Notes from the Vice Chairperson." - "Notes from the Vice Chairperson." - "Notes from the Vice Chairperson." - "Notes from the Vice Chairperson." - 


Clusters in range of year 1980 - 1995
2564 titles in this range
Top terms per cluster 1980 - 1995:
Cluster 0: "Semantics Based Transaction Management Techniques for Replicated Data." - "Efficient Management of Replicated Data." - "Temporal Data Management." - "Implementation of Information System Design Specifications: A Performance Perspective." - "Efficient Main Memory Data Management Using the DBGraph Storage Model." - 
Cluster 1: "An Entity-Relationship Algebra." - "Semantics of Query Languages for the Entity-Relationship Model." - "Entity-Generating Schema Transformations for Entity-Relationship Models." - "Towards a Semantic View of an Extended Entity-Relationship Model." - "Deductive Entity-Relationship Modeling." - 
Cluster 2: "Distributed Query Processing." - "Query Processing for Temporal Databases." - "Processing of Multiple Queries in Distributed Databases." - "Experiences with Distributed Query Processing." - "Distributed Query Processing Optimization Objectives." - 
Cluster 3: "Spatial Database Access Methods." - "An Implementation and Performance Analysis of Spatial Data Access Methods." - "Access Methods for Multiversion Data." - "On the Data Model and Access Method of Summary Data Management." - "A Data Model and Access Method for Summary Data Management." - 
Cluster 4: "Data Models and Languages for Databases." - "A Temporal Model and Query Language for ER Databases." - "A Homogeneous Relational Model and Query Languages for Temporal Databases." - "A Data Model and Query Language for EXODUS." - "A Graphical Query Language Based on an Extended E-R Model." - 
Cluster 5: "System/K: A Knowledge Base Management System." - "Research in Knowledge Base Management Systems." - "A Data/Knowledge Base Management Testbed and Experimental Results on Data/Knowledge Base Query and Update Processing." - "Evolution of Knowledge Bases." - "Efficient Management of Transitive Relationships in Large Data and Knowledge Bases." - 
Cluster 6: " Object-Oriented Database System." - "Queries in Object-Oriented Databases." - "A Model of Queries for Object-Oriented Databases." - "A Query Model for Object-Oriented Databases." - "Object-Oriented Database Systems." - 
Cluster 7: "Research Directions for Distributed Databases." - "Research Directions in Object-Oriented Database Systems." - "Research Directions in Image Database Management (Panel)." - "Research Issues in Spatial Databases." - "Database Research at the IBM Almaden Research Center." - 
Cluster 8: "Constraint-Based Query Evaluation in Deductive Databases." - "On Semantic Query Optimization in Deductive Databases." - "Constraint-Based Reasoning in Deductive Databases." - "A Rule-based Language for Deductive Object-Oriented Databases." - "Implementation of the CORAL Deductive Database System." - 
Cluster 9: "Efficient Transitive Closure Algorithms." - "Hybrid Transitive Closure Algorithms." - "A Performance Study of Transitive Closure Algorithms." - "A Generalized Transitive Closure for Relational Queries." - "Optimization of Generalized Transitive Closure Queries." - 
Cluster 10: "Concurrency Control for Relational Databases." - "A Unified Concurrency Control Algorithm for Distributed Database Systems." - "Concurrency Control for Distributed Real-Time Databases." - "Parallelism and Concurrency Control Performance in Distributed Database Machines." - "Semantic Concurrency Control in Object-Oriented Database Systems." - 
Cluster 11: "The Performance of Locking Protocols in Distributed Databases." - "An Integrated Real-Time Locking Protocol." - "A Dynamic Tree-Locking Protocol." - "Multiversion Query Locking." - "Locking Protocols for Concurrency Control in Real-time Database Systems." - 
Cluster 12: "Approaches to Design of Real-Time Database Systems." - "Issues and Approaches to Design of Real-Time Database Systems." - "Time and Databases." - "The Integration of Rule Systems and Database Systems." - "Parallel Database Systems: The Future of High Performance Database Systems." - 


Clusters in range of year 1990 - 2005
5006 titles in this range
Top terms per cluster 1990 - 2005:
Cluster 0: "A Relational Object Model." - "Transforming Conceptual Data Models into an Object Model." - "A Database Model for Object Dynamics." - "Mapping a Version Model to a Complex-Object Data Model." - "Using a Meta Model to Represent Object-Oriented Data Models." - 
Cluster 1: "Querying high-dimensional data in single-dimensional space." - "Fast Nearest Neighbor Search in High-Dimensional Space." - "On Optimizing Nearest Neighbor Queries in High-Dimensional Data Spaces." - "A Cost Model For Nearest Neighbor Search in High-Dimensional Data Space." - "The X-tree : An Index Structure for High-Dimensional Data" - 
Cluster 2: "A Query Model for Object-Oriented Databases." - "Querying Object-Oriented Databases." - "A Model for Active Object Oriented Databases." - ""Part" Relations for Object-Oriented Databases." - "Object Placement in Parallel Object-Oriented Database Systems." - 
Cluster 3: "Temporal Data Model." - "Query Processing for Temporal Databases." - "A Temporal Algebra for an ER-Based Temporal Data Model." - "Temporal Objects for Spatio-Temporal Data Models and a Comparison of Their Representations." - "Temporal Database System Implementations." - 
Cluster 4: "Scientific Workflow Management by Database Management." - "Multimedia Database Management Systems." - "The MARIFlow Workflow Management System." - "Data Management for Large Rule Systems." - "Transaction Management on Replicated Data." - 
Cluster 5: "Tree-Based Access Methods for Spatial Databases: Implementation and Performance Evaluation." - "Database Mining: A Performance Perspective." - "Concurrency Control: Methods, Performance, and Analysis." - "The Performance of a Multiversion Access Method." - "Distributed Optimistic Concurrency Control Methods for High-Performance Transaction Processing." - 
Cluster 6: "Minimizing Detail Data in Data Warehouses." - "Data Organization and Access for Efficient Data Mining." - "Simulation data as data streams." - "Data Warehouse Maintenance under Concurrent Schema and Data Updates." - "A Data Structure for Representing Aggregate Data." - 
Cluster 7: "Processing Top N and Bottom N Queries." - "A Graph Query Language and Its Query Processing." - "Dynamic Query Optimization and Query Processing in Multidatabase Systems." - "XML Query Processing and Optimization." - "XML Query Processing." - 
Cluster 8: "Modeling Multidimensional Databases." - "Querying Multidimensional Databases." - "A Logical Approach to Multidimensional Databases." - "OLAP, Relational, and Multidimensional Database Systems." - "Multidimensional Data Modeling for Complex Data." - 
Cluster 9: "Performance and Scalability of Client-Server Database Architectures." - "Client-Server Paradise." - "Performance Tradeoffs for Client-Server Query Processing." - "Persistent Client-Server Database Sessions." - "Client-Server Optimization for Multimedia Document Exchange." - 
Cluster 10: "Rules in Database Systems." - "What's Next in XML and Databases?" - "The Integration of Rule Systems and Database Systems." - "Databases and the Web: What's in it for Databases? (Panel)." - "Active Database Systems: Triggers and Rules For Advanced Database Processing." - 
Cluster 11: "Virtual Database Technology." - "Virtual Database technology." - "Database Technologies for E- Commerce." - "Database Technology for Internet Applications (Abstract)." - "Data Mining and Personalization Technologies." - 
Cluster 12: "Information rules." - "Dissemination-Based Information Systems." - "Web Information Retrieval." - "Image Information Systems: Where Do We Go From Here?" - "P2P Information Systems." - 
Cluster 13: "Spatial Joins and R-trees." - "Parallel Processing of Spatial Joins Using R-trees." - "Efficient Processing of Spatial Joins Using R-Trees." - "Spatial Join Indices." - "Spatial Hash-Joins." - 


Clusters in range of year 2000 - 2015
5859 titles in this range
Top terms per cluster 2000 - 2015:
Cluster 0: "-trees." - "Out From Under the Trees." - "-tree." - "Put a Tree Pattern in Your Algebra." - "The Bw-Tree: A B-tree for new hardware platforms." - 
Cluster 1: "Relational data sharing in peer-based data management systems." - "Data Management in the Cloud." - " queries on temporal data." - "A Data Model and Data Structures for Moving Objects Databases." - "Data Management in the Social Web." - 
Cluster 2: "Efficient Query Processing in Large Traffic Networks." - "Distributed processing of continuous join queries using DHT networks." - "Efficient Skyline Query Processing on Peer-to-Peer Networks." - "Answering graph pattern queries using views." - "Efficient Temporal Join Processing Using Indices." - 
Cluster 3: "Sampling-based estimators for subset-based queries." - "Pattern-Based Query Answering." - "Scalable ontology-based information systems." - "Logic-based Web Information Extraction." - "Ontology-Based Information Tailoring." - 
Cluster 4: "XML Query Processing and Optimization." - "Towards Semantic Query Optimization for XML Databases." - "Rule-based multi-query optimization." - "Cost-Based Object Query Optimization." - "Rank-Aware Query Processing and Optimization." - 
Cluster 5: "Towards scalable data integration under constraints." - "Constraint databases: A tutorial introduction." - "Query reformulation with constraints." - "Constraints for Semi-structured Data and XML." - "Efficient processing of graph similarity queries with edit distance constraints." - 
Cluster 6: "Privacy-Preserving Top-K Queries." - "Privacy-Preserving Data Mining." - "Privacy Preserving Joins." - "Privacy-preserving data publishing." - "Privacy-preserving indexing of documents on the network." - 
Cluster 7: "What's Next in XML and Databases?" - "XML-Based Applications Using XML Schema." - "XML Query Processing." - "XML schema." - "Updates on XML documents and schemas." - 
Cluster 8: "Querying Semantic Data on the Web?" - "Ruby on semantic web." - "Data Management Issues on the Semantic Web." - "Semantic Caching of Web Queries." - "The Grid: An Application of the Semantic Web." - 
Cluster 9: "A general and efficient algorithm for "top" queries." - "Memory-efficient algorithms for spatial network queries." - "Efficient search algorithm for SimRank." - "Efficient Algorithms for Mining Outliers from Large Data Sets." - "Efficient Rewriting Algorithms for Preference Queries." - 
Cluster 10: "Data Stream Query Processing." - "Optimization of Data Stream Processing." - "Continuous Queries over Data Streams." - "Unifying the Processing of XML Streams and Relational Data Streams." - " queries on uncertain streams." - 
Cluster 11: "T-Time: Threshold-Based Data Mining on Time Series." - "Scaling and time warping in time series querying." - "Querying time-series streams." - "Indexing Multidimensional Time-Series." - "Threshold Similarity Queries in Large Time Series Databases." - 
Cluster 12: "Using XML to Build Efficient Transaction-Time Temporal Database Systems on Relational Databases." - "Top-k Query Processing in Uncertain Databases." - "Efficient Processing of Top-k Queries in Uncertain Databases." - "A Database Approach to Quality of Service Specification in Video Databases." - "Querying Graph Databases." - 
Cluster 13: "Probabilistic pattern queries over complex probabilistic graphs." - "Query ranking in probabilistic XML data." - "Efficient query evaluation on probabilistic databases." - "Query evaluation over probabilistic XML." - "Efficient Top-k Query Evaluation on Probabilistic Data." - 
Cluster 14: "Evaluating Proximity Relations Under Uncertainty." - 


Clusters in range of year 2010 - 2025
5513 titles in this range
Top terms per cluster 2010 - 2025:
Cluster 0: "Efficient Skyline Computation in MapReduce." - "Towards efficient SimRank computation on large networks." - "Efficient computation of trade-off skylines." - "Efficient Computation of G-Skyline Groups (Extended Abstract)." - "SimRank computation on uncertain graphs." - 
Cluster 1: "Why-query support in graph databases." - "Query languages for graph databases." - "Query Games in Databases." - "Robust and Memory-Efficient Database Fragment Allocation for Large and Uncertain Database Workloads." - "Learning Path Queries on Graph Databases." - 
Cluster 2: "Efficient structural graph clustering: an index-based approach." - "Index-based Solutions for Efficient Density Peak Clustering (Extended Abstract)." - "Incremental semi-supervised clustering ensemble for high dimensional data clustering." - "Efficient Structural Clustering in Large Uncertain Graphs." - "An Efficient Framework for Exact Set Similarity Search Using Tree Structure Indexes." - 
Cluster 3: "Scalable top-k spatial keyword search." - "Keyword Search on Temporal Graphs." - "Top-k keyword search over probabilistic XML data." - "Keyword-based search and exploration on databases." - "Scalable keyword search on large data streams." - 
Cluster 4: "Continuous data cleaning." - "Cleaning uncertain data for top-k queries." - "A tool for Internet-scale cardinality estimation of XPath queries over distributed semistructured data." - "Contextual Data Cleaning." - "A gossip-based approach for Internet-scale cardinality estimation of XPath queries over distributed semistructured data." - 
Cluster 5: "Joint Entity Resolution." - "The Entity Name System: Enabling the web of entities." - "Entity Resolution with crowd errors." - "Parallel Progressive Approach to Entity Resolution Using MapReduce." - "Incremental entity resolution on rules and data." - 
Cluster 6: "Analysis and detection of low quality information in social networks." - "Parallel trajectory similarity joins in spatial networks." - "Efficient Bottom-Up Discovery of Multi-scale Time Series Correlations Using Mutual Information." - "Interaction-Aware Arrangement for Event-Based Social Networks." - "LoCEC: Local Community-based Edge Classification in Large Online Social Networks." - 
Cluster 7: "Skyline queries, front and back." - "Multiple-Query Optimization of Regular Path Queries." - "Provenance-Aware Query Optimization." - " queries on temporal data." - "Exploiting the query structure for efficient join ordering in SPARQL queries." - 
Cluster 8: "Top-k graph pattern matching over large graphs." - "Graph similarity search on large uncertain graph databases." - "Time Traveling in Graphs using a Graph Database." - "From user graph to Topics Graph: Towards twitter followee recommendation based on knowledge graphs." - "Finding top-k similar graphs in graph databases." - 
Cluster 9: "Main-memory database systems." - "Analytics on Fast Data: Main-Memory Database Systems versus Modern Streaming Systems." - "Towards a task-based search and recommender systems." - "Online Data Partitioning in Distributed Database Systems." - "Embarrassingly scalable database systems." - 
Cluster 10: "Real-Time Data Management for Big Data." - "Data-Less Big Data Analytics (Towards Intelligent Data Analytics Systems)." - "Data quality: The other face of Big Data." - "Distance-Based Data Mining over Encrypted Data." - "Workload management for Big Data analytics." - 
Cluster 11: "Efficient Continuous Multi-Query Processing over Graph Streams." - "Robust query processing." - "Robust query processing." - "Linked Data query processing." - "Robust distributed stream processing." - 
Cluster 12: "Ranking queries on uncertain data." - "Ranking for data repairs." - "Probabilistic ranking over relations." - "Distributed Similarity Joins over Top-K Rankings." - "Towards an Efficient Ranking of Interval-Based Patterns." - 
Cluster 13: "Scaling Up Subgraph Query Processing with Efficient Subgraph Matching." - "Efficient distributed subgraph similarity matching." - "An Indexing Framework for Efficient Visual Exploratory Subgraph Search in Graph Databases." - "Structure-preserving subgraph query services." - "Workload-Aware Subgraph Query Caching and Processing in Large Graphs." - 
Cluster 14: "Link prediction in graph streams." - "Learning-based Query Performance Modeling and Prediction." - "Destination prediction by sub-trajectory synthesis and privacy protection against such prediction." - "Link prediction across networks by biased cross-network sampling." - "Scalable Temporal Latent Space Inference for Link Prediction in Dynamic Social Networks (Extended Abstract)." - 
Cluster 15: "Metric all-k-nearest-neighbor search." - "K-Nearest Neighbor Temporal Aggregate Queries." - "Distributed in-memory processing of All K Nearest Neighbor queries." - "Privacy preserving group nearest neighbor queries." - "Privacy Preserving Group Nearest Neighbor Search." - 
\end{verbatim}

\subsection{Bespreking van de resultaten}
 
We hebben k-means 6 keer uitgevoerd, telkens op periodes van 15 jaar gaande van 1960-1975 t.e.m. 2010-2025. We zien dat er in de eerste twee tijdsperiodes relatief weinig artikels aanwezig waren (95 en 348 artikels respectievelijk), wat het in de eerste plaats al onwaarschijnlijker maakt dat er zeer sterk onderscheidbare clusters zouden te vinden zijn. Ook zaten hier veel titels zoals ``Editor's Notes'' en ``Title, Contents'' die niet interessant waren voor deze opdracht. Deze zijn er dan ook uitgefilterd met een blacklist. Hierdoor zijn nog wat entries weggevallen uit het tijdsinterval 1960-1990 want deze waren hier voornamelijk aantwezig. Deze zijn niet allen uitgefilterd, zo is er in het interval 1960-1975 nog steeds een cluster aanwezig die een ``Call for Papers'' representeerd en dus niet echt relevant is. We hebben dit ook geprobeerd op te lossen door custom stopwords toe te voegen, maar dit deed niet veel en kregen we soms lege titels. Wij waren van mening dat soms wat minder relevante entries beter zijn dan lege. \\

De 3 laatste tijdsperiodes (1990-2005 t.e.m. 2010-2025) tonen een beter representatief resultaat. Ook het aantal vereiste clusters is hier iets groter, met 14, 15 en 16, en dus ook evenveel verschillende topics. Om er een voorbeeld uit te halen: cluster 1 uit periode 2000-2015 lijkt vooral artikels te bevatten over data management, wat op het topic data management lijkt te wijzen voor deze cluster. \\
Een neveneffect dat we opnieuw waarnemen is dat er ook clusters worden gevormd op basis van features die niet topic gerelateerd zijn. Een voorbeeld is cluster 0 uit de periode 2000-2015. Deze cluster lijkt te zijn gevormd op basis van de feature dat alle titles woorden bevatten die met een liggend streepje zijn verbonden. Dit heeft uiteraard niets met data mining te maken. Dit is ook een fout die met wat data cleaning mogelijks voorkomen had kunnen worden. 

\section{Conclusie}
Uit de resultaten valt op dat de clustering in principe wel goed werkt. Het is duidelijk dat het programma correct artikels gaat samen clusteren op basis van tekstuele features in de titels. Echter, het is moeilijk features uit de titels te extracten die effectief aan de semantiek van de artikels gerelateerd zijn en niet louter over tekstuele eigenschappen gaan (zoals het voorkomen van liggende streepjes, voorkomen van woordgroepen als "Editor's Notes.", etc..). We denken hierdoor dat het gebruiken van titels van artikels niet het meest optimale onderdeel van een artikel is om te gebruiken voor het clusteren. Het zou beter zijn moest de input data bijvoorbeeld voor elk artikel een aantal tags bevatten over waar het artikel over gaat. Dit zou wellicht zorgen voor een veel accuratere clustering voor de opgestelde probleemstelling. Een ander alternatief zou zijn om features uit de volledige inhoud van de tekst van de artikels te halen, en niet enkel uit de titels. Echter zijn in de DBLP dataset die we gebruikt hebben noch tags noch de effectieve inhoud van de artikels beschikbaar en konden we deze hypothese dan ook niet uittesten. 



\end{document}
